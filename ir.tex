\documentclass[•]{article}
\usepackage{indentfirst}
\usepackage[usenames,dvipsnames]{color}
\usepackage[dvipdfm]{hyperref} 
\usepackage{enumerate}



\begin{document}

\section{topics in this field and corresponding readings}
\cite{CS598CXZreadings}

\subsection{Early history papers}
Getting the exact details right isn't so important as understanding the contributions of those early researchers and how their ideas have influenced the progress of the field.

\subsection{core topic}
At a high-level, there are three major core IR topics:

\begin{itemize}
\item Evaluation
\item Retrieval Models
\item Efficiency and Scalability. 
\end{itemize}


These topics are selected for two reasons: 

First, they represent the most important core topics in IR and are at the foundation of the modern search engines (i.e., ad hoc retrieval technologies). While IR research has made many contributions in other topics such as text categorization, clustering, summarization, and information filtering, research on these other topics tend to overlap a lot with research in other fields, especially machine learning, natural language processing, and data mining. In contrast, the selected core topics better represent unique contributions made by IR researchers. 

Second, while these topics are still active research topics today, research on these topics has now reached a "mature" state in that the research results on these topics so far likely will still represent the state of the art in the near future. In contrast, other important topics such as personalized search, query intent analysis, user interface, and various IR applications have not yet resulted in "stable" technologies, thus knowledge on these topics likely will become out of date soon. Your general goal in reading these materials should be to understand precisely the major techniques and research results, so that you can have a solid knowledge background on these core IR topics.

\subsection{frontier topics}


\section{RetMod}
todo: chapter 4 of \cite{Ponte1998}!

\subsection{prob}
\subsubsection{regression model and learning to rank}

\subsubsection{doc gen(1976)}

\subsubsection{query gen(1998 \cite{Ponte1998} \cite{Ponte1998a} i.e. langmod)}
***chapter 4 of \cite{Ponte1998} must read!

does not specify how weight of term is calculated

diff with Salton's VSM: q and d are not same, they are fundamentally diff (see p92 of \cite{Ponte1998a})

ref: \cite{Ponte1998} and \cite{Ponte1998a} and \cite{Fuhr1992}

\section{Q refinement}

\subsection{todo}
citing chain: \cite{Lv2011} -- [31]-- $>$ \cite{Zhai2001} --[10]--$>$ \cite{Ponte1998}

what is pseudo feedback?

see \cite{Lv2011} for RF methods for prob models(see subsection "relation with RM")

\subsection{essence}
guess user's intent? 

\subsection{aim}
*** to improve result of direct retrieval(i.e. search for one time)

so we can choose to refine the query

\subsection{relation with the whole retrieval approach}
see section 3 of \cite{Zhai2001}: the relation of method of using KL divergence to compute relevance value and the whole retrieval approach

\subsection{formulation of the problem}
\subsubsection{background}
problem:

of synonym in the q

how to solve it:

\begin{itemize}
\item user manually refine it
\item se help user to refine it
\subitem unsupervised: with feed back
\subitem supervised: without
\subitem global --- q expansion, etc.
\subitem local --- rf, etc.
\end{itemize}


\subsection{method A: refine manually(e.g. structured query---q op etc.??)}

\subsection{method B: se help user to refine it}

\subsubsection{supervised: with feedback}


\subsubsection{unsupervised: without feedback}
\begin{itemize}
\item global
\item local see \$5.3 of \cite{Ponte1998}
\subitem LCA
\subitem local feedback(pseudo feedback?)
\end{itemize}

\subsection{relation with RM}
\cite{Lv2011}
\begin{itemize}
\item for vsm:
rocchio
\item for prob model(lm, etc): select expansion terms primaily based on RSJ weight. (global method?)
\subitem for lm: KL divergence
\end{itemize}
\subsection{ref}
ItoIR(Cambridge) Chap9

\cite{Lv2011}, \cite{Zhai2001}

Zhai's courseware(\cite{CS598CXZreadings}, etc.)

Zhai's book

\section{paper reading}
\subsection{\cite{White2007}}
\subsubsection{1. intro}
general problem:

how to enable human to better utilize the machine specialized for searching(e.g. search engine)

\subsection{\cite{Zhai2001}}

\subsection{\cite{Ponte1998} and \cite{Ponte1998a}}
question: as it is early paper, make sure of what affect it makes on the following research!

\section{ref}


\bibliographystyle{plain}
\bibliography{library}

\end{document}